%---------------------------------------------------------------------------%
%->> Backmatter
%---------------------------------------------------------------------------%

\chapter[致谢]{致\quad 谢}\chaptermark{致\quad 谢}% syntax: \chapter[目录]{标题}\chaptermark{页眉}
\thispagestyle{noheaderstyle}% 如果需要移除当前页的页眉
%\pagestyle{noheaderstyle}% 如果需要移除整章的页眉

在中科院软件所互联网实验室攻读硕士研究生的三年即将结束,这段时间里,我不仅锻炼了自己的科研能力,更学会了团队的合作能力,这一切都要感谢各位老师和同学的帮助。

首先需要感谢的是我的导师王青研究员。王老师无论是从科学研究,工程实践
还是生活上都给予了我很大的帮助。王老师治学严谨、在科研领域具有丰富的经验并且待
人温和。通过和王老师的讨论交流,我不仅学习到了严谨的科研方法,还感受到一个优秀的科研工作者和教师所具备的责任感和奉献精神。在此祝愿王老师身体健康、阖家幸福。

感谢石琳老师在我科研实验以及论文写作中的悉心指导,为论文工作提出很多有价值的改进意见,强化了我的科研习惯以及论文写作能力,帮助我顺利的完成硕士期间的工程实践和科研工作。同时也要感谢王俊杰老师和王丹丹老师,她们在我的研究生期间提出了很多宝贵的建议,并在我的学习工作上给予了很大的帮助。

感谢李明阳师兄、王亚文师兄和陈磊师兄在研究生期间对于工程项目和科研工作上对我的帮助,在工程项目中,我们通力合作,一起解决了项目中遇到的各种困难,在这期间大大提高了我的工程能力和团队合作能力;在科研工作中,感谢师兄们给予的宝贵经验,帮助我更好的完成科研工作。感谢胡渊喆师兄、李守斌师兄在工作、生活中给予的帮助。感谢互联网实验室所有的师兄师姐在这三年以来对我的帮助,让我在这三年里面学习到很多。在此希望实验室发展越来越好,产出更多的科研成果。

感谢皇甫幼峰同学硕士期间一起工作、学习和讨论交流,感谢舍友吴杉同学、谭浩同学和王新刚同学三年间的鼓励和陪伴。

感谢家人和朋友的爱护和支持,使我顺利度过硕士期间生活和科研中遇到的困难和挫折。

最后感谢参与论文评审和答辩工作的各位老师以及答辩秘书,你们辛苦了。


\chapter{作者简历及攻读学位期间发表的学术论文与研究成果}

% \textbf{本科生无需此部分}。

\section*{作者简历}

2013年9月——2017年6月,在华中科技大学软件学院获得学士学位。

2017年9月——2020年7月,在中国科学院大学软件研究所攻读硕士学位。

\section*{已发表(或正式接受)的学术论文:}

{
\setlist[enumerate]{}% restore default behavior
\begin{enumerate}[nosep]
    \item Lin Shi*, \textbf{Mingzhe Xing*}, Mingyang Li, Yawen Wang, Shoubin Li, Qing Wang. Detection of Hidden Feature Requests from Massive Chat Messages via Deep Siamese Network . In 42nd International Conference on Software Engineering (ICSE’20), October 5-11, 2020, Seoul, Republic of Korea.
\end{enumerate}
}

\section*{投稿中(或在评审中)的学术论文:}

{
\setlist[enumerate]{}% restore default behavior
\begin{enumerate}[nosep]
    \item Lin Shi, Mingyang Li, \textbf{Mingzhe Xing}, Yawen Wang, Qing Wang, Xinhua Peng, Weimin Liao, Guizhen Pi, Peisheng Li. HyFinder: A Hybrid Function Extractor for Automated Requirement Analysis.投稿至FSE2020(International Symposium on Foundations of Software Engineering)
    \item Mingyang Li, Yawen Wang, \textbf{Mingzhe Xing}, Lin Shi, Qing Wang. Semi-Supervised Character-level Data Function Point Recognition with Low Labeled Resource.投稿至ICSME2020(International Conference on Software Maintenance and Evolution)
\end{enumerate}
}


\section*{参加的研究项目及获奖情况:}

{
\setlist[enumerate]{}% restore default behavior
\begin{enumerate}[nosep]
    \item 国家自然科学基金青年项目:“面向社交媒体的需求智能发现和分析方法”(项目编号:61802374),2019.1-2021.12
    \item 国家自然科学基金重点项目:“软件生命期数据组织、分析及应用研究”(项目编号:61432001),2015.1-2019.12
    % \item 国家科学基金项目:“众测环境下测试报告的智能筛选方法研究”(项目编号:61602450),2017.1-2019.12
    \item 国家重点研发计划项目:信息产品及科技服务集成化众测服务平台研发与应用(项目编号:2018YFB1403400),2019.7-2022.6
    \item “**银行智能软件研发效能研究项目”,2018.8-2020.8
\end{enumerate}
}

\cleardoublepage[plain]% 让文档总是结束于偶数页,可根据需要设定页眉页脚样式,如 [noheaderstyle]
%---------------------------------------------------------------------------%
