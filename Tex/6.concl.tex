\chapter{总结与展望}
\section{论文工作总结}
本文中,我们提出了一种使用孪生网络从大规模聊天信息中进行特征请求对话识别的新方法{\tool}。在{\tool}中,我们结合两个相同结构、共享参数的基于BiLSTM的孪生上下文敏感对话模型去学习对话之间的相似性,而不是传统的对单个对话进行分类。我们在一个从三个开源项目中的大规模聊天记录里面采样的1,035条对话数据上评估我们的模型效果。实验结果表明我们的方法在模型效果上相比两个句子级别的特征请求分类方法和四个文本分类模型有较高的提升 ,并取得了88.52\%,88.50\%和88.51\%的平均精确度、召回率、F1值。我们通过跨项目实验证明了{\tool}也具有较好的泛化能力,能将学习的特征请求相关模式泛化到其他项目领域。试验结果表明我们的方法可以有效地从大规模聊天信息中识别特征请求。另外,在四个文本分类方法中,我们观察到朴素贝叶斯可以取得最好的效果。

\section{未来研究工作}
经过全文论证和总结,本文提出的基于孪生网络从大规模聊天信息中识别特征请求对话方法有以下需要改进和进一步开展的工作:
\begin{enumerate}
    \item 我们准备将NLP摘要技术集成到我们的方法中,可以从冗长的聊天记录信息中作出简短摘要,这对开发者而言可以减少阅读特征请求对话的时间成本。
    \item 另外,我们准备扩展我们的工作,即不仅识别特征请求对话,并且使用精心设计的结构化模式对自然语言文本的对话进行结构化表示和信息抽取。
    \item 考虑到识别的特征请求对话中存在一定的重复特征请求,我们接下来将使用重复意图检测方法对识别到的大量特征请求对话进行筛选以减少开发者的工作。
\end{enumerate}