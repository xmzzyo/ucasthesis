\chapter{总结与展望}
\section{论文工作总结}
本文提出了一种使用孪生网络从大规模聊天信息中识别隐藏的需求对话的新方法{\tool}。本文的主要工作及贡献有:
\begin{enumerate}
    \item 本文是首个针对对话级别进行隐藏需求挖掘的工作,其适合于当前在线聊天平台在开发者之间越来越受欢迎、开发者倾向于在聊天中讨论需求的现状。
    \item 为解决对话数据标准困难的问题,{\tool}通过结合两个相同结构、共享参数的孪生上下文敏感对话模型{\dm}去学习对话之间的相似性,而不是传统的对单个对话进行分类的方法。
    \item 本文在三个开源项目上验证了{\tool}的效果,证明本文提出的方法在模型效果上相比两个句子级别的需求识别方法和四个文本分类模型有较高的提升,并具有较好的泛化能力,能将学习的需求对话相关模式泛化到其他项目领域。
    \item 本研究开源了标注的数据以及代码可供复现以及研究。
\end{enumerate}

因此,本文提出的{\tool}可以有效地从大规模聊天信息中识别隐藏的需求,进而为发布团队减少大量的重复工作。

另外,本文在针对实验验证方面存在的威胁以及解决方法有:
\begin{enumerate}
    \item 由于本研究仅在三个开源项目上进行了实验,并没有在其他项目上进行验证,因此可能存在数据集选取的随机性。针对此问题,本文进行了项目间泛化能力实验,实验结果证明了{\tool}较强的项目间泛化能力。
    \item 由于对话解耦工具的解耦结果会对后续实验结果产生影响,因此可能会导致实验结果错误和偏差。为了减少此误差,本文使用目前效果最好的对话解耦模型。
    \item 由于在手工标注对话数据集时可能会带来标注误差,因此在实验评估时可能会对结果造成偏差。对此,本工作构建了两个标注团队,并指定统一的标注规则以期团队间对标注结果达成一致来还缓解此问题。
\end{enumerate}






\section{未来研究工作}
经过全文论证和总结,本文提出的基于孪生网络从大规模聊天信息中识别隐藏需求对话方法有以下需要改进和进一步开展的工作:
\begin{enumerate}
    \item 接下来准备将NLP摘要技术集成到本文方法中,其可以从冗长的聊天记录信息中作出简短摘要,这对开发者而言可以减少阅读需求对话的时间成本。
    \item 另外准备扩展本文工作,即使{\tool}不仅可以识别需求对话,并且可以使用经过精心设计的结构化模式对自然语言文本的对话进行结构化表示和信息抽取。
    \item 考虑到识别的需求对话中存在一定的重复需求,本工作接下来将使用重复意图检测等方法对识别到的大量需求对话进行筛选,以减少开发者重复的工作。
\end{enumerate}