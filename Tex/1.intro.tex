% !TEX root = ..\thesis.tex
\chapter{引言}\label{chap:introduction}

\section{研究背景及意义}

近来,随着软件过程管理、互联网模式的不断发展与更新,软件应用的数量以及规模的不断扩大,软件应用的迭代也不断加快。如何迅速及时地发现高质量的软件需求,并对其进行准确的提取成为开源社区以及业界需要解决的问题 \cite{Saur2006Review}。


IEEE软件工程中定义需求为:用户解决问题或达到目标所需的条件和能力;系统或系统部件为满足合同、标准、规范或其它正式规定文档所需具有的条件和能力;以上条件和能力的文档说明 \cite{Board2002IEEE}。软件需求可分为业务需求、用户需求、功能性需求、非功能性需求、设计约束、系统约束等。


需求抽取和发现是指收集准备建立的系统和正在使用的系统的信息,并从这些信息当中提取用户和系统需求 \cite{Vlas2011A},需求的获取应注意三个方面:应获取的信息,使用的信息来源,采用的获取技术。为了收集到全面完整的信息,需求获取可以通过对现有系统分析、与潜在用户和购买者交流、讨论等方式来导出软件系统的需求。需求发现阶段的信息源包括已有文件、系统信息持有者以及类似系统的相关内容。需求的获取有多种渠道,比如可以通过市场调研的方式,从市场走向分析以及从产品走向分析,确定用户的相关需求和边际需求;也可以开发一个或者多个系统模型或原型来帮助用户更好的表达系统的需求 \cite{唐中君基于},这样有利于系统分析员了解所要描述的系统的功能;需求人员还可以把自己作为系统的最终用户,审视该系统并提出问题,但这需要但这需要需求人员具有比用户更多的应用领域和过程管理方面的知识,并具有良好的想象能力;为了确定系统应该具有的功能,需求人员通过提出问题,用户回答,直接询问用户想要的是什么样的系统;观察是另一种需求获取的方法,通过观察用户执行其现行的任务和过程,或者通过观察他们如何操作与所期望的新系统相关的现有系统,了解系统运行的环境,特别是了解要建的新系统和现存系统、过程以及工作方法之间的必须进行的交互;另一种方式时通过小组会的方式,可以举行客户和开发人员的联席会议,与客户组织的一些代表共同开发需求;我们还可以通过提炼的方式获取需求 \cite{孙挺2002基于},提炼方法是针对已经有了部分需求文档的情况,复审技术文档,例如有关需求的陈述功能和性能目标的陈述,系统规约接口标准,硬件设计文档,以及ConOps文档,并提出相关信息。在特定的环境中,每项技术都有其自己的优点和不足,在实施上述任何一项技术时,都可以辅以其他方法,比如原型构造,在举行小组会时可以使用原型,方便人员之间的交流。依据需求工程人员的技能和产品、合同的实际情况,往往需要组合使用这些技术来开发初始需求。执行需求发现这项活动的人,其技能水平对这项活动的成功也具有巨大的影响。

在当今许多开源项目和工业软件中,开发者大量使用交流平台, 比如邮件,问题追踪,聊天等 \cite{panichella2014developers}表达他们的去定义软件的特征的需求\cite{fitzgerald2006transformation}。在这些数据中包含的信息已经被研究者用来构建推荐系统,比如

理解这些需求可以为开源项目提供启发和见解。但是手工对自然语言需求进行分析是十分耗时的,甚至会产生错误的。对自然语言需求自动进行分析会带来巨大的好处。


\section{研究内容和创新点}

因为用户和开发者之间、公司开发者团队内部之间等通常最先在开放式聊天平台显式或潜在地提出需求,里面会存在着大量的原始需求。聊天数据是需求的一个典型来源,一般会要求开发者在会话期间记录需求。并且聊天数据量一般较大,蕴含时间、用户角色等对需求十分重要的信息,我们可以从中挖掘出大量的用户的原始需求,并使用结构化的方式对其进行解释性说明,对通过传统方式发现需求的方式起到了启发、补充等作用。观察分割后的会话可看出,对话中需求的表达形式因角色、上下文的不同而不同,有些显式地表达出清晰地需求,有些较为隐式地或者在交谈过程中逐步确定需求,因此,根据传统地基于规则地语义识别方法很难以及不能完整地识别并抽取需求,对此,我们使用深度学习基于对话形式进行需求识别和抽取。


本课题将主要从开放式聊天平台的数据出发,因聊天数据的规模较大、需求表述不规范、不明确、需求稀疏等问题,我们利用需求工程以及自然语言处理、机器学习等技术抽取出需求会话,并对其进行推荐,以达到从源头出发发现需求,快速及时发现需求,可以减少开发人员在聊天过程中对需求的记录,避免丢失重要的关于需求的信息,并对传统需求发现方式起到启发,补充的目的。

\section{论文组织结构}



\section{本章小结}

