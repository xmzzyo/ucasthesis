% !TEX root = ..\thesis.tex
\chapter{引言}\label{chap:introduction}

\section{研究背景及意义}

近来,随着软件过程管理、互联网模式的不断发展与更新,软件应用的数量以及规模的不断扩大,软件应用的迭代也不断加快。如何迅速及时地发现高质量的软件需求,并对其进行准确的提取成为开源社区以及业界需要解决的问题 \cite{Saur2006Review}。


IEEE软件工程中定义需求为:用户解决问题或达到目标所需的条件和能力;系统或系统部件为满足合同、标准、规范或其它正式规定文档所需具有的条件和能力;以上条件和能力的文档说明 \cite{Board2002IEEE}。软件需求可分为业务需求、用户需求、功能性需求、非功能性需求、设计约束、系统约束等。


需求抽取和发现是指收集准备建立的系统和正在使用的系统的信息,并从这些信息当中提取用户和系统需求 \cite{Vlas2011A},需求的获取应注意三个方面:应获取的信息,使用的信息来源,采用的获取技术。为了收集到全面完整的信息,需求获取可以通过对现有系统分析、与潜在用户和购买者交流、讨论等方式来导出软件系统的需求。需求发现阶段的信息源包括已有文件、系统信息持有者以及类似系统的相关内容。需求的获取有多种渠道,比如可以通过市场调研的方式,从市场走向分析以及从产品走向分析,确定用户的相关需求和边际需求;也可以开发一个或者多个系统模型或原型来帮助用户更好的表达系统的需求 \cite{唐中君基于},这样有利于系统分析员了解所要描述的系统的功能;需求人员还可以把自己作为系统的最终用户,审视该系统并提出问题,但这需要但这需要需求人员具有比用户更多的应用领域和过程管理方面的知识,并具有良好的想象能力;为了确定系统应该具有的功能,需求人员通过提出问题,用户回答,直接询问用户想要的是什么样的系统;观察是另一种需求获取的方法,通过观察用户执行其现行的任务和过程,或者通过观察他们如何操作与所期望的新系统相关的现有系统,了解系统运行的环境,特别是了解要建的新系统和现存系统、过程以及工作方法之间的必须进行的交互;另一种方式时通过小组会的方式,可以举行客户和开发人员的联席会议,与客户组织的一些代表共同开发需求;我们还可以通过提炼的方式获取需求 \cite{孙挺2002基于},提炼方法是针对已经有了部分需求文档的情况,复审技术文档,例如有关需求的陈述功能和性能目标的陈述,系统规约接口标准,硬件设计文档,以及ConOps文档,并提出相关信息。在特定的环境中,每项技术都有其自己的优点和不足,在实施上述任何一项技术时,都可以辅以其他方法,比如原型构造,在举行小组会时可以使用原型,方便人员之间的交流。依据需求工程人员的技能和产品、合同的实际情况,往往需要组合使用这些技术来开发初始需求。执行需求发现这项活动的人,其技能水平对这项活动的成功也具有巨大的影响。

在当今许多开源项目和工业软件中,开发者大量使用交流平台, 比如邮件,问题追踪,聊天等 \cite{panichella2014developers}表达他们的去定义软件的特征的需求\cite{fitzgerald2006transformation}。在这些数据中包含的信息已经被研究者用来构建推荐系统,比如

理解这些需求可以为开源项目提供启发和见解。但是手工对自然语言需求进行分析是十分耗时的,甚至会产生错误的。对自然语言需求自动进行分析会带来巨大的好处。

最近的研究报告表明,在线聊天的使用越来越流行,并且在软件开发中起着越来越重要的作用,在某些情况下已经取代了电子邮件。开发人员正在转向公共工作聊天平台,例如Slack、IRC、HipChat、Gitter和Freenode,以在上面分享观点和有趣的见解、讨论如何解决缺陷以及将来要实现的功能。
尽管开发人员在与其他开发人员进行交流时会表达其所需的功能,但是由于在线聊天的开放性和拥挤性,使这些特征请求对话很快就会被新收到的消息淹没。通常情况下,如果在线聊天中讨论的特征请求未被记录在文档中,则可能会被淹没或忽略。如图\ref{fig:motivation}所示,以AngularJS项目的聊天消息为例,开发人员P和F在在线聊天中发布了他们的问题。最初,他们的意图是寻求其他开发人员的帮助,以寻求可行的解决方案。与其他开发人员聊天之后,他们意识到现有系统无法按照他们想要的方式运行。然后,他们的意图从寻求解决方案转变为特征请求。P请求“交互性复选框的ng-true-value / ng-false-value之类的东西”,F表示“angular cli可以将我的所有服务放入服务文件夹”。
\begin{figure}[htbp]
    \centering
    \includegraphics[width=0.70\textwidth]{Img/motivation.pdf}
    \bicaption{AngularJS项目的一个聊天信息例子,特征请求淹没在大量聊天历史中}{Example chat message from AngularJS project, where requests to desired features are buried in the massive chat history.}
    \label{fig:motivation}
\end{figure}

在这项工作中,我们将包含对新特征/增强特征的请求的对话视为特征请求对话。实际上,发布团队监视各种沟通渠道,以获取与下一个发布版本可能相关的多个信息源。如果发布团队可以从聊天消息中确认那些隐藏的特征请求,则他们的下一个发布计划可能会通过考虑实现更多特征请求而最大化利益相关者的满意度。

从大量的聊天消息中检索有价值信息的自动化挖掘技术非常需要从大量用户那里收集全面的特征请求,这有助于需求的确定和发布计划,进而促进软件开发的成功。尽管聊天消息可能很庞大,并且随着时间的推移加入了特征请求,但是由于以下困难,要挖掘大量的聊天消息还是非常具有挑战性的。
\begin{enumerate}
    \item \textbf{对话级分析:}分析来自聊天消息的对话不同于常规的文本挖掘任务,因为它在理解一个句子时需要考虑对话范围内的上下文信息。因此,关于句级别特征请求发现的现有研究不能直接用于此任务。例如,句子“我们需要添加垂直导航栏选项“被句级别技术分类为特征请求。但是,当在在线聊天中发出时,后续对话指出现有功能可以以替代方式满足该请求。此外,逐句的检测结果是不准确的,因为其在聊天消息中将大量离题句子识别为特征请求,例如,“我真的需要恢复我的编程技能“,”我想喝点咖啡和饼干“。
    \item \textbf{极其昂贵的标注:}聊天消息通常规模很大。在大量聊天消息之间查找特征请求对话,就像在大海捞针一样。由于庞大的语料库和少量真实数据,标注聊天消息中的特征请求对话极其昂贵,只有少数带标签的聊天消息被分类为特征请求类型。如何最大程度地利用少量标记数据来对未标记聊天消息进行准确分类成为一个关键问题。
    \item \textbf{耦合和噪声数据:}聊天消息通常是大规模的,并且是包含涉及广泛主题的非正式对话。两个或多个开发人员彼此同步地进行交互,其中他们的对话在聊天记录中很大程度上被耦合。此外,在聊天消息中存在噪声文本,例如重复消息和离题消息,这些消息不提供任何有价值的信息。耦合和噪声数据给分析和解释对话带来了困难。
\end{enumerate}

\section{研究内容和创新点}

因为用户和开发者之间、公司开发者团队内部之间等通常最先在开放式聊天平台显式或潜在地提出需求,里面会存在着大量的原始需求。聊天数据是需求的一个典型来源,一般会要求开发者在会话期间记录需求。并且聊天数据量一般较大,蕴含时间、用户角色等对需求十分重要的信息,我们可以从中挖掘出大量的用户的原始需求,并使用结构化的方式对其进行解释性说明,对通过传统方式发现需求的方式起到了启发、补充等作用。观察分割后的会话可看出,对话中需求的表达形式因角色、上下文的不同而不同,有些显式地表达出清晰地需求,有些较为隐式地或者在交谈过程中逐步确定需求,因此,根据传统地基于规则地语义识别方法很难以及不能完整地识别并抽取需求,对此,我们使用深度学习基于对话形式进行需求识别和抽取。


本课题将主要从开放式聊天平台的数据出发,因聊天数据的规模较大、需求表述不规范、不明确、需求稀疏等问题,我们利用需求工程以及自然语言处理、机器学习等技术抽取出需求会话,并对其进行推荐,以达到从源头出发发现需求,快速及时发现需求,可以减少开发人员在聊天过程中对需求的记录,避免丢失重要的关于需求的信息,并对传统需求发现方式起到启发,补充的目的。

在这项工作中,我们是首先在对话级别上进行分析的,该技术旨在自动检测聊天消息中发布的隐藏特征请求。我们提出了一种名为FRMiner的新方法,该方法可以通过深层的孪生网络从聊天消息中检测特征请求对话。为了更好地理解对话级别的上下文信息,我们首先基于双向LSTM(BiLSTM)结构构建了一个上下文感知对话模型,该模型可以深入学习正向和反向对话的上下文信息。受到通过利用不足的标记资源来建立预测模型的少样本学习技术的启发,我们将把单个对话分类到其类别的传统文本分类任务转换为确定两个对话属于相同还是不同类的任务。因此,我们将上下文感知对话模型与孪生网络相结合,以学习一对对话之间的相似性,而不是单一对话类别的方式。特征请求对话的预测结果可以基于相似性预测及其配对对话中观察到的类别来推断。为了评估我们提出的方法,我们标注了三个流行的开源项目中的1,035个对话。实验结果表明,我们的方法明显优于两个句子级别分类器和四个传统文本分类方法,其平均精度,召回率和F1值分别为88.52%,88.50%和88.51%。结果证实,我们的方法可以有效地检测聊天消息中的隐藏特征请求,从而可以自动方式从大量用户收集全面的需求。

本文主要贡献主要有以下几方面:
\begin{enumerate}
    \item  我们是第一个提出从大量聊天消息中检测隐藏特征请求的,这些请求可以帮助全面的需求收集。
    \item 我们引入了一种解决方案,该解决方案可以通过结合孪生网络来基于有限的标注数据有效地预测特征请求对话,从而大大减轻了对有监督数据进行标注的负担。
    \item 我们在三个活跃的开源项目中评估了我们的方法,并进行了实验比较,结果表明所提出的方法优于现有的研究和四个文本分类方法。
    \item 可公开访问的数据集和源代码,以促进我们的研究及其在其他情况下的重用。
\end{enumerate}



\section{论文组织结构}



\section{本章小结}

